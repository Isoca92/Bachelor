%
% Zusammenfassung/Abstract
%
\section*{Zusammenfassung}
Die �berpr�fung von chemischen Verbindungen auf ihre molekulare �hnlichkeit ist eine h�ufig genutzte Methode in der modernen Wirkstoffentwicklung. Durch die Bestimmung der molekularen �hnlichkeit k�nnen Vorhersagen zur molekularen Interaktion getroffen werden. So k�nnen anhand von bereits bekannten Verbindungen neue Verbindungen gefunden werden, die sich in ihrer biologischen Aktivit�t �hneln. Es gibt bereits eine Reihe von Verfahren, die erlauben R�ckschl�sse auf die �hnlichkeit zweier chemischen Verbindungen zu ziehen. Zur Analyse von Molek�lmengen kann das Konzept der molekularen �hnlichkeit auf chemische Muster �bertragen werden. Chemische Muster, zum Beispiel durch SMARTS-Ausdr�cke repr�sentiert, beschreiben strukturelle Eigenschaften oder funktionelle Gruppen. Anhand dieser Darstellung von bestimmten Merkmalen werden ganze Molek�lmengen beschrieben, die dieses chemische Muster beinhalten.\\
Es wurde ein Verfahren entwickelt, welches SMARTS-Ausdr�cke miteinander vergleicht und auf Teilmengen-Relationen �berpr�ft. Realisiert wurde dieses Verfahren durch die Generierung von sogenannten Fingerprints, die es erm�glichen die komplexen Atom- und Bindungs-Repr�sentationen der SMARTS-Sprache zu vergleichen.

\section*{Abstract}
An estimation of the molecular similarity of chemical compounds is a popular approach in the modern drug-discovery. To predict the molecular interactions of as new compounds the approach of the molecular similarity makes it possible to draw conclusions from known connections. So the biological activity could be similar to known compounds. There are a set of methods to calculate the similarity of molecules. To analyze a set of molecules the concept can be transferred to chemical patterns. A set of molecules can be represented as a chemical Pattern written as a SMARTS-pattern. A SMARTS-pattern represents a set of molecules by discribing characteristical structrures or functional groups.\\
An approach to compare chemical patterns and draw conclusions about the existing subset relationships. This approache was implemented by using the concept of fingerprints, which were calculated fot each atom and bound in a chemical pattern, to allow to compare those complex SMARTS-pattern.

\newpage
%
% EOF
%