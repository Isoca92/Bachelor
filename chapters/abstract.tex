%
% Zusammenfassung/Abstract
%
\section*{Zusammenfassung}
Chemische Muster beschreiben strukturelle Eigenschaften oder funktionelle Gruppen, die in Molek�len enthalten sein k�nnen. Anhand dieser Darstellung von bestimmten chemischen Merkmalen werden ganze Molek�lmengen beschrieben. Chemische Muster k�nnen zum Beispiel mit der SMARTS-Sprache ausgedr�ckt werden.\\
Vorhersagen zur molekularen Interaktion durch den Vergleich mit bereits bekannten Verbindungen zu treffen, ist eine h�ufig genutzte Methode in der modernen Wirkstoffentwicklung. So k�nnen neue Verbindungen gefunden werden, die sich in ihrer biologischen Aktivit�t �hneln. Der Vergleich erfolgt dabei meist durch eine Analyse auf molekulare �hnlichkeit. Um Molek�lmengen, beschrieben durch chemische Muster, zu vergleichen, k�nnen sie auf Teilmengen-Relationen gepr�ft werden. Solch ein Vergleich erm�glicht es R�ckschl�sse auf eine vorliegende Hierarchie der chemischen Muster zu ziehen oder Redundanzen ausschlie�en zu k�nnen. Ein Vergleich von chemischen Mustern ist, durch die hohe Komplexit�t dieser Beschreibungen bislang nicht realisiert worden.\\
Es wurde ein Verfahren entwickelt, welches SMARTS-Ausdr�cke miteinander vergleicht und auf Teilmengen-Relationen �berpr�ft. Realisiert wurde dieses Verfahren durch die Generierung von sogenannten Fingerprints, die es erm�glichen die komplexen Atom- und Bindungsrepr�sentationen der SMARTS-Sprache zu vergleichen. Abschlie�end wurde das Verfahren evaluiert und Experimente mit realen Daten durchgef�hrt, um einen m�glichen Anwendungsbereich zu pr�sentieren.

\section*{Abstract}
A chemical pattern describes characteristical structures or functional groups that occur in molecules. A set of molecules can be represented as a chemical pattern written as a SMARTS-pattern.\\
Predictions of molecular interactions of chemical compounds are a popular approach in modern drug discovery. By that the biological activity will presumably be similar to known compounds. An often used method for comparison is the analysis of the molecular similarity. To compare chemical patterns it is possible to calculate subset relationships. This could clarify a hierarchy or point out redundancies. Until now the high complexity of SMARTS pattern prevent a comparison between chemical patterns.\\
An approach is to compare chemical patterns and draw conclusions about the existing subset relationships. This approach was implemented by using the concept of fingerprints, which were calculated for each atom and bound in a chemical pattern respectively, to allow the comparison of those complex SMARTS pattern. In conclusion the method was evaluated and applied on real data to present a possible scope of applications.
\newpage
%
% EOF
%